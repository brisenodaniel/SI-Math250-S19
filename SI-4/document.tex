\documentclass[12pt]{article}
\usepackage{mathtools}
\usepackage{amsfonts}
\usepackage{amssymb}
\usepackage{amsthm}
\usepackage{enumitem}
\usepackage{stmaryrd}
\usepackage[letterpaper, total={6in, 10in}]{geometry}
\date{}
\author{}
\title{SI 4: Function Composition, Nested Induction, and Midterm}
\begin{document}
	
	\maketitle
	\section{Function Composition}
	\begin{enumerate}
		\item Provide a function who's domain and co-domain can only be $\mathbb{N}$
		\item Consider a function $f$ that returns its input with decimal point truncated, such that $f(0.1) = 0$, $f(\pi) = 3$, $f(2) =2$. What would the function's ``signature" $f: A \to B$ be? (i.e. what would its domain an co-domain be)?
		\item Is this function one-to-one (injective)? Is it onto (surjective)?
		\item Define this function as a composition of other functions (hint: it may be useful to come up with your own definition of the division function first, think back to elementary school and remainders)
	\end{enumerate}
	\section{Double Induction}
	Consider the following function:
	\begin{align*}
		f(1,1) &= 2\\
		f(d^{\curvearrowright}, k) &= f(d,k) + 2(d+k)\\
		f(d, k^{\curvearrowright}) &= f(d,k) + 2(d + k-1)
	\end{align*}
	\textbf{Prove by induction:}
	\begin{align*}
	\forall{m,n} \in \mathbb{N}\:\:|\:\: m,n>1\\
	f(m,n) &= (m+n)^2 - (m+n) -2n + 2
	\end{align*}
\end{document}